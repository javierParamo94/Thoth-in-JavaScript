\documentclass[a4paper,11pt,oneside]{memoir}

% Castellano
\usepackage[spanish,es-tabla]{babel}
\selectlanguage{spanish}
\usepackage[utf8]{inputenc}
\usepackage{placeins}

\RequirePackage{booktabs}
\RequirePackage[table]{xcolor}
\RequirePackage{xtab}
\RequirePackage{multirow}

% Links
\usepackage[colorlinks]{hyperref}
\hypersetup{
	allcolors = {red}
}

% Ecuaciones
\usepackage{amsmath}

% Rutas de fichero / paquete
\newcommand{\ruta}[1]{{\sffamily #1}}

% Párrafos
\nonzeroparskip


% Imagenes
\usepackage{graphicx}
\newcommand{\imagen}[2]{
	\begin{figure}[!h]
		\centering
		\includegraphics[width=0.9\textwidth]{#1}
		\caption{#2}\label{fig:#1}
	\end{figure}
	\FloatBarrier
}

\newcommand{\imagenflotante}[2]{
	\begin{figure}%[!h]
		\centering
		\includegraphics[width=0.9\textwidth]{#1}
		\caption{#2}\label{fig:#1}
	\end{figure}
}



% El comando \figura nos permite insertar figuras comodamente, y utilizando
% siempre el mismo formato. Los parametros son:
% 1 -> Porcentaje del ancho de página que ocupará la figura (de 0 a 1)
% 2 --> Fichero de la imagen
% 3 --> Texto a pie de imagen
% 4 --> Etiqueta (label) para referencias
% 5 --> Opciones que queramos pasarle al \includegraphics
% 6 --> Opciones de posicionamiento a pasarle a \begin{figure}
\newcommand{\figuraConPosicion}[6]{%
  \setlength{\anchoFloat}{#1\textwidth}%
  \addtolength{\anchoFloat}{-4\fboxsep}%
  \setlength{\anchoFigura}{\anchoFloat}%
  \begin{figure}[#6]
    \begin{center}%
      \Ovalbox{%
        \begin{minipage}{\anchoFloat}%
          \begin{center}%
            \includegraphics[width=\anchoFigura,#5]{#2}%
            \caption{#3}%
            \label{#4}%
          \end{center}%
        \end{minipage}
      }%
    \end{center}%
  \end{figure}%
}

%
% Comando para incluir imágenes en formato apaisado (sin marco).
\newcommand{\figuraApaisadaSinMarco}[5]{%
  \begin{figure}%
    \begin{center}%
    \includegraphics[angle=90,height=#1\textheight,#5]{#2}%
    \caption{#3}%
    \label{#4}%
    \end{center}%
  \end{figure}%
}
% Para las tablas
\newcommand{\otoprule}{\midrule [\heavyrulewidth]}
%
% Nuevo comando para tablas pequeñas (menos de una página).
\newcommand{\tablaSmall}[5]{%
 \begin{table}
  \begin{center}
   \rowcolors {2}{gray!35}{}
   \begin{tabular}{#2}
    \toprule
    #4
    \otoprule
    #5
    \bottomrule
   \end{tabular}
   \caption{#1}
   \label{tabla:#3}
  \end{center}
 \end{table}
}

%
% Nuevo comando para tablas pequeñas (menos de una página).
\newcommand{\tablaSmallSinColores}[5]{%
 \begin{table}[H]
  \begin{center}
   \begin{tabular}{#2}
    \toprule
    #4
    \otoprule
    #5
    \bottomrule
   \end{tabular}
   \caption{#1}
   \label{tabla:#3}
  \end{center}
 \end{table}
}

\newcommand{\tablaApaisadaSmall}[5]{%
\begin{landscape}
  \begin{table}
   \begin{center}
    \rowcolors {2}{gray!35}{}
    \begin{tabular}{#2}
     \toprule
     #4
     \otoprule
     #5
     \bottomrule
    \end{tabular}
    \caption{#1}
    \label{tabla:#3}
   \end{center}
  \end{table}
\end{landscape}
}

%
% Nuevo comando para tablas grandes con cabecera y filas alternas coloreadas en gris.
\newcommand{\tabla}[6]{%
  \begin{center}
    \tablefirsthead{
      \toprule
      #5
      \otoprule
    }
    \tablehead{
      \multicolumn{#3}{l}{\small\sl continúa desde la página anterior}\\
      \toprule
      #5
      \otoprule
    }
    \tabletail{
      \hline
      \multicolumn{#3}{r}{\small\sl continúa en la página siguiente}\\
    }
    \tablelasttail{
      \hline
    }
    \bottomcaption{#1}
    \rowcolors {2}{gray!35}{}
    \begin{xtabular}{#2}
      #6
      \bottomrule
    \end{xtabular}
    \label{tabla:#4}
  \end{center}
}

%
% Nuevo comando para tablas grandes con cabecera.
\newcommand{\tablaSinColores}[6]{%
  \begin{center}
    \tablefirsthead{
      \toprule
      #5
      \otoprule
    }
    \tablehead{
      \multicolumn{#3}{l}{\small\sl continúa desde la página anterior}\\
      \toprule
      #5
      \otoprule
    }
    \tabletail{
      \hline
      \multicolumn{#3}{r}{\small\sl continúa en la página siguiente}\\
    }
    \tablelasttail{
      \hline
    }
    \bottomcaption{#1}
    \begin{xtabular}{#2}
      #6
      \bottomrule
    \end{xtabular}
    \label{tabla:#4}
  \end{center}
}

%
% Nuevo comando para tablas grandes sin cabecera.
\newcommand{\tablaSinCabecera}[5]{%
  \begin{center}
    \tablefirsthead{
      \toprule
    }
    \tablehead{
      \multicolumn{#3}{l}{\small\sl continúa desde la página anterior}\\
      \hline
    }
    \tabletail{
      \hline
      \multicolumn{#3}{r}{\small\sl continúa en la página siguiente}\\
    }
    \tablelasttail{
      \hline
    }
    \bottomcaption{#1}
  \begin{xtabular}{#2}
    #5
   \bottomrule
  \end{xtabular}
  \label{tabla:#4}
  \end{center}
}



\definecolor{cgoLight}{HTML}{EEEEEE}
\definecolor{cgoExtralight}{HTML}{FFFFFF}

%
% Nuevo comando para tablas grandes sin cabecera.
\newcommand{\tablaSinCabeceraConBandas}[5]{%
  \begin{center}
    \tablefirsthead{
      \toprule
    }
    \tablehead{
      \multicolumn{#3}{l}{\small\sl continúa desde la página anterior}\\
      \hline
    }
    \tabletail{
      \hline
      \multicolumn{#3}{r}{\small\sl continúa en la página siguiente}\\
    }
    \tablelasttail{
      \hline
    }
    \bottomcaption{#1}
    \rowcolors[]{1}{cgoExtralight}{cgoLight}

  \begin{xtabular}{#2}
    #5
   \bottomrule
  \end{xtabular}
  \label{tabla:#4}
  \end{center}
}


















\graphicspath{ {./img/} }

% Capítulos
\chapterstyle{bianchi}
\newcommand{\capitulo}[2]{
	\setcounter{chapter}{#1}
	\setcounter{section}{0}
	\chapter*{#2}
	\addcontentsline{toc}{chapter}{#2}
	\markboth{#2}{#2}
}

% Apéndices
\renewcommand{\appendixname}{Apéndice}
\renewcommand*\cftappendixname{\appendixname}

\newcommand{\apendice}[1]{
	%\renewcommand{\thechapter}{A}
	\chapter{#1}
}

\renewcommand*\cftappendixname{\appendixname\ }

% Formato de portada
\makeatletter
\usepackage{xcolor}
\newcommand{\tutor}[1]{\def\@tutor{#1}}
\newcommand{\course}[1]{\def\@course{#1}}
\definecolor{cpardoBox}{HTML}{E6E6FF}
\def\maketitle{
  \null
  \thispagestyle{empty}
  % Cabecera ----------------
\noindent\includegraphics[width=\textwidth]{cabecera}\vspace{1cm}%
  \vfill
  % Título proyecto y escudo informática ----------------
  \colorbox{cpardoBox}{%
    \begin{minipage}{.8\textwidth}
      \vspace{.5cm}\Large
      \begin{center}
      \textbf{TFG del Grado en Ingeniería Informática}\vspace{.6cm}\\
      \textbf{\LARGE\@title{}}
      \end{center}
      \vspace{.2cm}
    \end{minipage}

  }%
  \hfill\begin{minipage}{.20\textwidth}
    \includegraphics[width=\textwidth]{escudoInfor}
  \end{minipage}
  \vfill
  % Datos de alumno, curso y tutores ------------------
  \begin{center}%
  {%
    \noindent\LARGE
    Presentado por \@author{}\\ 
    en Universidad de Burgos --- \@date{}\\
    Tutor: \@tutor{}\\
  }%
  \end{center}%
  \null
  \cleardoublepage
  }
\makeatother

\newcommand{\nombre}{Francisco Javier Páramo Arnaiz} %%% cambio de comando

% Datos de portada
\title{Conversión de la aplicación docente Thoth a JavaScript}
\author{\nombre}
\tutor{Dr. Cesar Ignacio García Osorio y D. Álvar Arnaiz González}
\date{\today}

\begin{document}

\maketitle


\newpage\null\thispagestyle{empty}\newpage


%%%%%%%%%%%%%%%%%%%%%%%%%%%%%%%%%%%%%%%%%%%%%%%%%%%%%%%%%%%%%%%%%%%%%%%%%%%%%%%%%%%%%%%%
\thispagestyle{empty}


\noindent\includegraphics[width=\textwidth]{cabecera}\vspace{1cm}

\noindent  Dr. Cesar Ignacio García Osorio y D. Álvar Arnaiz González, profesores del departamento de Departamento de Ingeniería Civil, área de Lenguajes y Sistemas Informáticos.

\noindent Expone:

\noindent Que el alumno D. \nombre, con DNI 71293768-R, ha realizado el Trabajo final de Grado en Ingeniería Informática titulado Conversión de la aplicación docente Thoth a JavaScript. 

\noindent Y que dicho trabajo ha sido realizado por el alumno bajo la dirección del que suscribe, en virtud de lo cual se autoriza su presentación y defensa.

\begin{center} %\large
En Burgos, {\large \today}
\end{center}

\vfill\vfill\vfill

% Author and supervisor
\begin{minipage}{0.45\textwidth}
\begin{flushleft} %\large
Vº. Bº. del Tutor:\\[2cm]
D. Álvar Arnaiz González
\end{flushleft}
\end{minipage}
\hfill
\begin{minipage}{0.45\textwidth}
\begin{flushleft} %\large
Vº. Bº. del co-tutor:\\[2cm]
Dr. Cesar Ignacio García Osorio
\end{flushleft}
\end{minipage}
\hfill

\vfill

% para casos con solo un tutor comentar lo anterior
% y descomentar lo siguiente
%Vº. Bº. del Tutor:\\[2cm]
%D. nombre tutor


\newpage\null\thispagestyle{empty}\newpage




\frontmatter

% Abstract en castellano
\renewcommand*\abstractname{Resumen}
\begin{abstract}
La teoría de autómatas y lenguajes formales es uno de los campos más antiguos en el ámbito de la informática, que a pesar de los constantes avances, sus principales conceptos continúan siendo de referencia para el estudio teórico y práctico de la informática. Esto es la base sobre la que nació el mundo de las tecnologías que conocemos hoy en día.

Thoth es un antiguo proyecto de fin de grado que cuenta con varias versiones realizado por alumnos de la Universidad de Burgos y que sirve, precisamente, como herramienta de ayuda en la docencia de la teoría de autómatas y lenguajes formales. Este proyecto pretende llevar a Thoth a la web, sumando nuevas funcionalidades propias de este tipo de aplicaciones, como el registro de usuarios o inicio de sesión, que la harán más útil y completa.


Para ello se hace un estudio sobre las posibles tecnologías que ayudarán a conseguir el objetivo. De esta forma se determina como herramienta para lograr este cambio el <<framework>> denominado Google Web Toolkit, que consigue traducir parte del código escrito en Java a código JavaScript. GWT, según sus siglas, es una tecnología algo antigua pero muy útil para este proyecto, que combina código del lado del cliente y código del lado del servidor, cada uno con una funcionalidad, y que se comunican entre ellos vía RPC.

El trabajo fundamental es adaptar el código Java para que el framework logre interpretarlo y traduzca lo necesario para que la aplicación se ejecute en cualquier navegador. Se trata de aprender los entresijos de GWT, sacando partido de la dualidad cliente-servidor y conservando todo lo bueno, que es mucho, realizado en la aplicación de escritorio de Thoth.

\end{abstract}

\renewcommand*\abstractname{Descriptores}
\begin{abstract}
Palabras separadas por comas que identifiquen el contenido del proyecto Ej: servidor web, buscador de vuelos, android \ldots
\end{abstract}

\clearpage

% Abstract en inglés
\renewcommand*\abstractname{Abstract}
\begin{abstract}
A \textbf{brief} presentation of the topic addressed in the project.
\end{abstract}

\renewcommand*\abstractname{Keywords}
\begin{abstract}
keywords separated by commas.
\end{abstract}

\clearpage

% Indices
\tableofcontents

\clearpage

\listoffigures

\clearpage

\listoftables
\clearpage

\mainmatter
\capitulo{1}{Introducción}

Este proyecto nació con el objetivo de llevar Thoth \cite{garcia2007ensenanza}, un antiguo proyecto escrito en Java, a la web. Para ello se utilizarán tecnologías web, que puedan ser utilizadas en diferentes dispositivos haciéndolo accesible a todo el mundo. Con ayuda de la herramienta conocida como GWT o \emph{Google Web Toolkit} \footnote{\url{http://www.gwtproject.org/}} según su denominación inglesa, traduciré la aplicación a Javascript en el lado del cliente haciendo posible la utilización de la aplicación directamente desde un navegador, algo que antes era imposible.

Thoth \cite{garcia2007ensenanza}
 es un antiguo proyecto enfocado a la actividad docente y relacionado con los procesadores de lenguaje, que fue realizado por varios alumnos de la Universidad de Burgos como trabajo de fin de carrera. Esta aplicación cuenta con dos versiones  hasta la fecha y con otros desarrollos como Web Thoth \cite{jute2017}.

Una de las principales preguntas que nos podemos hacer al ver este proyecto es el por qué traducir la aplicación al lenguaje JavaScript. 
En primer lugar todos los navegadores actuales son capaces de interpretar el código escrito en JavaScript y soportan al menos alguna de las versiones ECMAScript \cite{ecma:versiones}, siendo la última la séptima edición disponible desde 2016. Con ayuda de la tecnología AJAX, se puede ejecutar la aplicación en el cliente, es decir, en el navegador de un usuario mientras se mantiene la comunicación asíncrona con el servidor en segundo plano.

Utilizo GWT para este proyecto como herramienta para transformar la aplicación de escritorio a entornos web. Pese a que actualmente no es el mejor \emph{framework} para desarrollar, sí es una de las base de otro mucho más moderno y potente. Hablo de Vaadin, que se encuentra más actualizado y con más bibliotecas para un aspecto visual más moderno, pero que para poder hacer disfrutar de su versión completa, se necesita una licencia bajo pago. La forma de trabajar con GWT es creando el código en Java y el compilador hará una traducción a los lenguajes JavaScript y HTML.

Una de las cuestiones más importantes de este \emph{frameworks} es que GWT no es capaz de admitir todas las bibliotecas de Java y por eso debimos reescribir el código adaptándolo a las capacidades de este \emph{framework}. Encontramos otras opciones de las que posteriormente hablaré, que son capaces de hacer la traducción completa de Thoth en Java a HTML5 de una forma muchísimo más sencilla y directa sin necesidad, si quiera, de esforzarse demasiado. Pero ese no es el objetivo de un trabajo de fin de grado, sino el de plantear un desafío suficientemente grande con el que el alumno pueda hacer uso del conjunto de muchos de los conocimientos aprendidos a lo largo de grado.
\capitulo{2}{Objetivos del proyecto}

El objetivo principal de este trabajo de final de grado consiste en transformar el proyecto previo de Thoth, que esta escrito en Java, a la tecnología web JavaScript. Esta conversión se llevará a cabo por medio de GWT, herramienta elegida posteriormente a un estudio previo. 

Consiste pues, en desarrollar una página web dinámica que sea capaz de hacer las mismas funcionalidades sobre gramáticas formales y sus algoritmos que en Thoth, poniendo a prueba mis propios conocimientos sobre Java así como otros lenguajes sobre tecnologías web como son HTML, CSS o JavaScript.

Pero el porque pretendemos pasar de una aplicación de escritorio a otra que se ejecuta en el navegador es básicamente porque no es necesario instalar o descargar material para poderla utilizar Thoth, simplemente algunos conocimientos básicos sobre gramáticas formales. Por lo tanto aporta una clara ventaja con respecto a aplicaciones denominadas <<de escritorio>> como es el Thoth original sobre el que nos basamos en este proyecto.

La conversión de Thoth a GWT es necesaria porque amplía las posibilidades de mejora de las anteriores versiones. Pretendemos aportar mayor funcionalidad debido a que al ser una aplicación web podemos hacer un registro de los usuarios, con inicio de sesión, registro de actividades, inicio de sesiones, etc. 

De hecho, uno de los puntos fuertes de este proyecto consiste en aprovechar la mayor parte de utilidad de las aplicaciones web. Aunque en las anteriores versiones la pretensión era sobre todo didáctica en materia relacionada con el estudio de los procesos del lenguaje, ahora a demás de eso también queremos llegar a poder saber la utilidad que se le da a Thoth. Para ello gracias al registro de información sobre los usuarios podremos saber cuando se ha registrado alguien, o iniciado sesión, que gramáticas ha usado y más funcionalidades que puedan surgir. Eso si, siempre manteniendo la privacidad del usuario.

Por último objetivo está el de experimentar con herramientas que no conocemos, y de esta forma ir aprendiendo a solucionar problemas nuevos. Al fin y al cabo lo que se trata de aprender en la universidad es a aprender a superar estos retos por uno mismo, aplicando las técnicas que se nos enseñan.
\capitulo{3}{Conceptos teóricos}

En mi experiencia como profesor de programación, diseño 3D y robótica para alumnos de primaria, lo primero que les enseñábamos era a definir la programación como el lenguaje de comunicación entre nosotros, los humanos, y los ordenadores o los robots. Todo esto pertenece a los conceptos más puramente teóricos sobre la computación y que es, en sí, la base de la informática. 

Pues bien, a \emph{grosso modo} el concepto es el mismo ahora. Como cada uno, hombre y máquina, <<habla>> un idioma diferente se debe establecer un lenguaje que sea la vía de comunicación entre ambos. 

Para poder entender todo lo que se va a tratar a continuación, debo explicar primero el concepto de \textit{autómata} o \textit{máquina abstracta}. Un autómata es un modelo matemático o un dispositivo teórico que recibe una cadena de símbolos como entrada y que al procesarla, genera un cambio de estado produciendo una salida determinada. Esta salida puede reconocer palabras y determinar si la entrada pertenece a un determinado lenguaje o no. En el símil anterior, un autómata es como el corrector, que determina si algo está bien escrito o no. 

Entonces ya sabemos que, para que haya comunicación entre un ordenador o robot y un humano, tiene que haber un lenguaje y un autómata. Pero también algo más: una gramática.

\section{¿Qué es una gramática y para qué sirve?}

Una gramática formal es un mecanismo para la generación de cadenas de caracteres que definen un determinado lenguaje formal, y que utiliza un conjunto de reglas de formación. Por lo tanto, podemos entenderlo dentro del concepto de las ciencias de la computación y la lógica matemática. Las cadenas de caracteres resultantes son a su vez <<bien formadas>> cuando pertenecen al lenguaje formal con el que se trabaja \cite{aho1986compilers}.

¿Y por qué es tan importante? Siguiendo con el ejemplo del principio, la gramática es la que va a determinar si lo que se introduce en el autómata es correcto o no. Un conjunto de reglas que nos indicará el por qué al juntar una serie de caracteres de una forma se van a poder entender.

Por otro lado, la denominación de la gramática formal, desde un punto de vista más formal, es una cuádrupla compuesta por:

\begin{itemize}
	\item Un alfabeto de \textbf{símbolos terminales} o tókenes denotado con la letra griega $\Sigma$.
	\item $\mathcal{N}$ que es un alfabeto formado por \textbf{símbolos no terminales}.
	\item Un alfabeto de \textbf{producciones} denominado $\mathcal{P}$.
	\item Y por último, un símbolo llamado \textbf{axioma} o símbolo inicial, denotado por $\mathcal{S}$ y tal que $\mathcal{S} \in \mathcal{N}$.
\end{itemize}

El alfabeto total que compone la gramática está formado, según lo anterior, por $\Sigma\cup\mathcal{N}$, es decir, por el conjunto de los símbolos terminales y no terminales.

Una \textbf{producción} tiene esta estructura y está compuesto por un par ordenado de cadenas. \[x \rightarrow y\] A la parte izquierda se la denomina antecedente y la derecha consecuente. A las producciones también se las denomina reglas de derivación.

Pongamos un ejemplo de una gramática simple y veamos de que está formada. Se suele utilizar la notación  \[x \rightarrow y\] \[z \rightarrow w\] para indicar una o varias producciones, en vez de \[(x, y) \in \mathcal{P} \] \[(z, w) \in \mathcal{P} \] siendo $\mathcal{P}$ el conjunto de producciones.

Por otro lado, si hay más de una producción que comience con el mismo elemento la notación sería de esta forma \[ x \rightarrow y | z | w\] en lugar de ser \[ x \rightarrow y\] \[x \rightarrow z\] \[x \rightarrow w\].

Cuando la gramática tiene una producción $x \rightarrow y$, se dice que $w$ \textbf{deriva directamente} de $v$ ( $v \Rightarrow w$) si $\exists z, u \in \Sigma^{*}$ tales que $v=zxu, w=zyu$ y $x\rightarrow y$.

Sea $\Sigma$ un alfabeto, un conjunto de producciones $\mathcal{P}$ y $v, w \in \Sigma^{*}$ decimos que $w$ \textbf{deriva} de $v$, si existen $u_{0}, u_{1}, ..., u_{n} \in \Sigma^{*}$ y una secuencia de transformaciones tales que:
 \[ v = u_{0} \Rightarrow u_{1}\] \[ u_{1} \Rightarrow u_{2}\] \[...\] \[ u_{n-1} \Rightarrow u_{n} = w\]

Esta secuencia de transformaciones se dice que es una \textbf{derivación} de longitud $n$. Se habla de lenguaje generado por una gramática $G$, denotado por $L(G)$, y se define como el conjunto de todas las sentencias de la gramática $G$.  \[L(G) = x\in \Sigma^{*}:\mathcal{S} \Rightarrow^{+}x\]
De esta manera se dice que dos gramáticas son equivalentes $G_{1} \equiv G_{2}$, si generan el mismo lenguaje, es decir, si $L(G_{1})=L(G_{2})$.

\section{Recursividad}

Por último, una gramática es recursiva en un símbolo no terminal $U$ cuando existe una forma sentencial de $U$ que contiene a $U$. \[U \Rightarrow^{+}xUy \textup{ donde } x,y\in (\Sigma \cup \mathcal{N})^{*} \]
Así pues la gramática será recursiva cuando lo sea para algún no terminal, si $x = \varepsilon$ la gramática es recursiva por la izquierda, y si $y = \varepsilon$ se dice que es recursiva por la derecha.

\section{Tipos de gramáticas }

Hay varios tipos de gramáticas según sus características \cite{aho1986compilers}.
\begin{itemize}
\item \textbf{Gramáticas de Chomsky}: las producciones tienen la forma \[u \rightarrow v \textup{ donde } u=xAy \in (\Sigma \cup \mathcal{N})^{+} \wedge A \in \mathcal{N} \wedge x, y, v \in (\Sigma \cup \mathcal{N})^{*}\]

Es posible demostrar que los lenguajes generados por una gramática de Chomsky se puede generar por un grupo más restringido llamadas gramáticas con estructura de frase, es decir, que ambas tienen la misma capacidad generativa. Las gramáticas de frase tienen la siguiente forma de producción:

\[xAy \rightarrow xvy \textup{ donde } x,y,v \in (\Sigma \cup \mathcal{N})^{*} \wedge A \in \mathcal{N}\]
\item \textbf{Gramaticas dependientes o sensibles al contexto}: Cuentas con las reglas de producción de la forma  \[xAy \rightarrow xvy \textup{ donde } x, y \in (\Sigma \cup \mathcal{N})^{*} \wedge A \in \mathcal{N} \wedge v \in (\Sigma \cup \mathcal{N})^{+}\]
En este tipo de gramáticas el significado de $A$ depende del contexto o de la posición en la frase. El contexto sería entonces $x$ e $y$. Además, la longitud de la parte derecha de las producciones es siempre mayor o igual que la de la parte izquierda.
\item \textbf{Gramáticas independientes del contexto}: las producciones de las gramáticas de este tipo son más restrictivas, de la forma: 
\[A \rightarrow v \textup{ donde } A \in \mathcal{N} \wedge v \in (\Sigma \cup \mathcal{N})^{*}\]
Como su propio nombre indica, el significado de $A$ es independiente de la posición en la que se encuentra. La mayor parte de los lenguajes de ordenador pertenecen a este tipo. Una característica importante es que las derivaciones obtenidas al utilizarse esta gramática se pueden representar utilizando árboles.

\item \textbf{Gramáticas regulares}: es el grupo más restringido. Tienen la forma: \[A \rightarrow aB \wedge A \rightarrow b \textup{ llamadas gramáticas regulares a derechas } \]
\[A \rightarrow Ba \wedge A \rightarrow b \textup{ llamadas gramáticas regulares a izquierdas } \]
\[ \textup{ donde }A, B \in \mathcal{N} \wedge a,b \in \Sigma  \]
Ambas son equivalentes. Existe también una generalización de este tipo de gramáticas denominadas lineales con reglas de la forma: \[A \rightarrow wB \wedge A \rightarrow v \textup{ lineales a derechas } \]
\[A \rightarrow Bw \wedge A \rightarrow v \textup{ lineales a izquierdas } \]
\[ \textup{ donde }A, B \in \mathcal{N} \wedge w,v \in \Sigma^{*}  \]
que son totalmente equivalentes a las regulares normales, pero en muchos casos su notación es más adecuada.
\end{itemize}

\section{Cifrado de contraseñas y seguridad}

El cifrado de la contraseña es una de las cuestiones más importantes para preservar la seguridad o intimidad del usuario. El por qué es muy simple. Los usuarios normalmente utilizamos las mismas contraseñas o parecidas para cualquier cuenta. Por lo tanto si hay alguien con acceso a la base de datos, podrá ver la contraseña utilizada por un determinado usuario, poniendo en peligro esa intimidad, no solo en este programa, sino, como ya se ha comentado antes, en las de otras cuentas.

Para evitar esto, el método más simple es el cifrar las contraseñas en la base de datos para que en el caso en el que alguien acceda a ella, en el campo <<contraseña>> no vea la contraseña real, sino el resultado del cifrado de esta. El método de cifrado que se emplea eneste proyecto el del algoritmo \emph{hash} que como veremos más adelante tiene unas características determinadas. 

\subsection{Funcionamiento del algoritmo hash}

Se trata de un algoritmo matemático con el que se transforma cualquier cantidad de datos en una serie de datos fija que funciona como una huella dactilar. Esto quiere decir que sea cual sea la cantidad de caracteres de entrada, la salida siempre será fija. Cumple además con dos premisas muy importantes para la seguridad: 

\begin{itemize}
\item No es reversible, no se puede descifrar por medio de funciones matemáticas y obtener el resultado antes de ser encriptado, sea cual sea la función utilizada (SHA-1, SHA-2 o MD 5 entre otras, como veremos a continuación).
\item Cuenta con la propiedad de que si la entrada cambia, aunque sea sólo en un bit, el \emph{hash} resultante será completamente distinto, como se puede ver en la ilustración \ref{fig:3.1}. En la imagen se puede apreciar que aunque la entrada tenga un mayor número de caracteres, la salida siempre será de 40 caracteres.

\end{itemize} 

\begin{figure}[h]
\centering
\includegraphics[width=0.99\textwidth]{funcion-hash}
\caption{Aplicación de la función hash a diferentes datos introducidos.}{\url{https://blog.kaspersky.com.mx/que-es-un-hash-y-como-funciona/2806/}}
\label{fig:3.1}
\end{figure}

Ahora bien, las acciones llevadas a cabo para preservar esa seguridad serían las siguientes: un usuario crea una cuenta en la aplicación, la contraseña se encripta y se almacena en la base de datos. Cuando el usuario trata de iniciar sesión y escribe la contraseña, esta se encripta y se compara el resultado con aquel que se ha guardado en la base de datos, si son iguales el usuario tendrá acceso; sino, se le requerirá que lo vuelva a intentar.

El problema principal de esto es que con los avances en temas de seguridad siempre hay asociados otros que tratan de <<romper>> esa seguridad y en este caso no iba a ser menos. Creo que es importante reconocer los peligros que hay asociados a aplicar este método, pero no deseo extenderme demasiado en este aspecto así que los mencionaré brevemente.


\begin{itemize}
\item Ataques de fuerza bruta. Consisten en utilizar diccionarios de palabras con contraseñas habituales e introducirlas hasta que alguna coincida. Se trata del ataque menos eficiente, pero el más difícil de evitar.
\item Tablas de búsqueda. Este tipo de ataque sí que supondría un grave problema para la seguridad en el cifrado con algoritmo \emph{hash}. Recordemos que las funciones \emph{hash} solo se pueden encriptar, no descifrar. Lo que hace este ataque es lo siguiente: cuenta con una tabla de contraseñas típicas y su cifrado \emph{hash} y las compara con los \emph{hash} introducidos. Para entender mejor este concepto hay incluso herramientas \emph{online} que pueden hacer este trabajo. Ver ilustración~\ref{fig:3.2}. También las hay que funcionan al revés. Introduces la contraseña que crees que puede usar alguien, la encripta, la compara con todas las de la base de datos y te dice si alguien la utiliza o no.
\end{itemize}

\begin{figure}[h]
\centering
\includegraphics[width=0.99\textwidth]{hash-cracker}
\caption{Ejemplo de ataque con tablas de búsqueda.}{Imagen sacada de \url{https://crackstation.net/}}
\label{fig:3.2}
\end{figure}

Además de estos, hay más métodos, casi todos basados en las tablas de búsqueda. Según estos ataques queda comprobado que la seguridad de este algoritmo depende en gran mediada de la contraseña que utilice el usuario. Cuanto más aleatoria y con más mezcla de caracteres mejor, ya que formará palabras que no se encuentran en los diccionarios o tablas y solo se podrá descubrir por medio de la fuerza bruta. Entonces ¿cómo hacer que las contraseñas sean más resistentes y solucionar este problema?

Se conoce como el cifrado hash con sal o semilla. Consiste en añadir un conjunto de caracteres aleatorios, agregarlos a la contraseña y una vez hecho esto, cifrarlo con la función hash. De esta manera se consigue que la contraseña sea mucho más aleatoria que la que inicialmente ha introducido el usuario. Así podemos ver como quedaría un intento de <<tablas de búsqueda>> con este tipo de cifrado se usa para contraseñas muy simples \ref{fig:3.3}. 

\begin{figure}[h]
\centering
\includegraphics[width=0.99\textwidth]{hash_salt-cracker}
\caption{Resultado de aplicar tablas de búsqueda al cifrado hash con semilla.} {Imagen sacada de \url{https://crackstation.net/}}
\label{fig:3.3}
\end{figure}

La primera de ellas corresponde con la clave <<123>> la segunda con la palabra <<contraseña>> y la tercera con la fecha <<16-6-17>>. Todas son fáciles, pero al añadirle una clave y cifrarlo todo, se vuelve mucho mas complejo. Se podría pensar que, como la clave que se le añade también se guarda en la base de datos sigue sin ser seguro. Partiendo del hecho de que no hay prácticamente nada seguro al cien por cien, lo que se consigue con esto es dificultar mucho el cálculo, ya que los dos ataques mencionados anteriormente (y varios más basados en estos dos) se ayudan de repositorios de contraseñas. Así, si quisieran descifrarlo, debería de coger el conjunto aleatorio, sumarle la posible contraseña y cifrarlo comparando los resultados sólo con ese \emph{hash} (o contraseña cifrada) ya que el \emph{salt} es único para cada \emph{hash}. Con este método se consigue que los cálculos para descifrar una contraseña sean bastante más costosos de una forma muy simple.

\capitulo{4}{Técnicas y herramientas}

Dentro de las diferentes herramientas que utilizaré para la realización de este trabajo, la más importante es aquella con la cual, realizaré la conversión a JavaScript. Es por ello esencial hacer una buena elección comparando y analizando la diferentes posibilidades a elegir.

Como principales herramientas para la conversión de Java a JavaScript he podido encontrar GWT (Google Web Toolkit), JSweet, WebSwing, Vaadin y DukeScript aunque también hay otras que descartamos por su poco relevancia o información.

\section{GWT}

Google Web Toolkit es un framework ámpliamente conocido por los desarrolladores web, entre otras cosas, gracias a ser de código abierto, de su gran utilidad y calidad además de ser completamente gratuito \footnote{\url{http://www.gwtproject.org/}}.
Contiene una SDK que proporciona un conjunto de APIs de Java que permiten el desarrollo de aplicaciones AJAX escritas en Java. Posteriormente compila el código en JavaScript ya optimizado dando robustez a la aplicación web.

Básicamente, permite a los desarrolladores compilar código JAVA en archivos JavaScript ya optimizados de forma autónoma, proporcionando así todas las ventajas de las aplicaciones escritas en este último lenguaje. 

GWT permite compartir código escrito en Java en la parte del servidor con código JavaScript en la parte del cliente lo que nos lleva pensar que la aplicación resultante será fiel a la idea inicial del Thoth. 

Nos decantamos por GWT porque, a parte de que supone un aprendizaje para mí como alumno, también es la base de otros frameworks de los que más tarde hablaré. Ha sido muy utilizado anteriormente y ahora está digamos que en decadencia. El soporte actual es mínimo y sobre todo en el tema visual anda algo anticuado. En el desarrollo del proyecto hemos llegado a ver  y probar <<bugs>> que según otros usuarios ya descubrieron hace un par de años.

Aun así, hay bastante información con la que he podido trabajar, y una comunidad grande, que aunque ahora se ha <<pasado>> a otros frameworks más actuales, han dejado huella y soluciones a muchos de los problemas con los que he trabajado.

\section{WebSwing}

En cuanto a esta herramienta, es algo diferente a las demás. WebSwing la descubrimos debido a una duda que nos surgió a al principio del proyecto. Y es que la aplicación Thoth original cuenta con muchísimos elementos de la biblioteca gráfica <<Swing>>, es digamos toda la estructura visual que utiliza. El problema surgió que con GWT no podemos hacer uso de ella ya que al ser algo visual debe ir en la parte del cliente y como ya hemos mencionado, en el cliente, que es donde se hace la traducción a JavaScript, las librerías de Java para desarrollar son muy limitadas. 

Pues bien al buscar una alternativa, descubrimos WebSwing. Se trata de un servidor web que permite la ejecución de aplicaciones que utilicen la biblioteca gráfica Swing desde el navegador, utilizando sólo HTML5. De esta forma toda la aplicación de Thoth se ejecutaría en el navegador conservando su aspecto de siempre y manteniendo las ventajas de una aplicación web.

En vez de utilizar JavaScript utiliza HTML5, cumpliendo además el objetivo principal del proyecto, que es llevar Thoth a la web.

La cuestión es que el utilizar esta herramienta no supone ningún reto como informático y facilitaría tanto el proyecto que este quedaría en nada más que unas simples mejoras de Thoth hechas con poco desarrollo. Por lo tanto la descartamos después de haberla probado.

\section{JSweet} 

JSweet\footnote{\url{http://http://www.jsweet.org/}}es básicamente un <<transpiler>> es decir un compilador que traduce un código en un lenguaje a otro lenguaje. Al igual que GWT esta orientado a objetos, que proporciona una programación segura gracias a que usa un sistema de <<tipado>> Java.

La diferencia fundamental con GWT es que al ser un <<transpiler>> hace una traducción directa entre Java y JavaScript posicionando el código a un lado o al otro del Cliente y el servidor. Esto, claramente, tiene sus ventajas y sus inconvenientes dependiendo del uso que se le quiera dar. 

porque no queriamos esto

\section{DukeScript}

Se define como una tecnología para la creación de aplicaciones Java <<multiplataforma>> que internamente hacen uso de tecnologías HTML5 y JavaScript para el renderizado.\footnote{\url{https://dukescript.com/}}
Al igual que en los casos anteriores <<sólo>> se necesita desarrollar la aplicación en Java para después transformarla. Y digo <<sólo>> porque eso es en la teoría ya que como hemos podido ver, y en parte es lógico, la traducción suele requerir, por lo menos, realizar ajustes del lenguaje para un buen funcionamiento.

DukeScript se centra sobre todo en el desarrollo de aplicaciones <<multi-plataforma>> llevadas a cabo en Java, más que en el paso de Java a JavaScript. Da la posibilidad de que alguien con conocimientos, digamoslo así, en Java pueda llevar a cabo un proyecto en lenguajes pensados para aplicaciones móviles o web. Esto no quita que se puedan realizar aplicaciones de escritorio con JavaScript.

---añadir más---

\section{Vaadin}

Vaadin en un <<framework>> de Java de código abierto, para crear aplicaciones web \footnote{\url{https://vaadin.com/home}}. Se programa en Java o cualquier otro lenguaje de JVM. Lo mas destacado de Vaadin es que esta construido sobre una base de GWT, por ello es una de las grandes alternativas a este último. La forma de trabajar con Vaadin es mediante el lenguaje Java e incorpora un lado cliente y otro servidor, el el cual irán las funcionalidades más complejas y su programación es dirigida por eventos. Es decir, hasta aquí es igual a GWT.

Las mejoras con respecto a GWT son varias, pero voy a mencionar solo aquellas que son más relevantes para este proyecto. Cuenta sobre todo con muchos elementos visuales, mejorados y con diseños más actuales. La parte visual es tan importante en Vaadin que incluyen un <<diseñador>> o <<designer>> en inglés, en le puggin de Eclipse que facilita mucho la creación de la parte visual ya que da la posibilidad de hacer el diseño de forma visual.

En realidad la mayor parte de los elementos visuales, menús, <<boards>>, diagramas estadísticos, iconos etc, están pensados sobre todo para un uso comercial orientado sobre todo para empresas. Por ello, el problema principal es que para poder hacer uso de su potencial se necesitan licencias de pago.

Aunque cuenta con un núcleo de elementos gratuitos y periodos de prueba también gratuitos, decidimos seguir nuestro camino, por llamarlo así, con GWT y hacerlo completamente de esta forma.

tecnicas de cifrado? 
de mantenimiento de session?
subrayado con HTML?
cifrado?
\capitulo{5}{Aspectos relevantes del desarrollo del proyecto}

En esta sección trataré de explicar aspectos más técnicos sobre el proyecto con el fin de entenderlo y poder continuar con su desarrollo en un posible futuro.

Las herramientas que utilizaré requieren de una instalación y configuración 
previa. Para ello voy a  tratar de ilustrar el cómo hacerlo.

\section{Instalación y configuración de GWT:}

Para poder trabajar con GWT debo descargar el SDK de GWT proporcionado en la página web oficial. Los requisitos previos para crear un aplicación web con Google Web Toolkit son básicamente dos: tener intalada la SDK de Java en su versión 1.6 o cualquiera superior a esta y tener instalado tambien Apache Ant o en su defecto Apache Maven.

Es fácil saber si cumplo ambos requisitos. Una vez descargado el SDK desde la página oficial,(poner página oficial??) accedo a la carpeta desde la consola de comandos, y ahí intento ejecutar el comando "webAppCreator". En caso de que la consola me devuelva un error en el que indica que Java no ha podido reconocerlo como un comando interno, quiere decir que no cumplo eso requisitos previos.

(Explicar como descargar e intalar el SDK de JAVA? y Apache Ant?? Confirurar el path del sistema, añadir una nueva variable etc.)

Posteriormente puedo proceder de varias formas y con diferentes plataformas. En mi caso he elegido la plataforma Eclipse sobre la que trabajaré en  su versión denominada Luna. 

GWT se instala en Eclipse como un "Plug-in" y para ello debo ir, dentro de Eclipse a añadir un nuevo software. (sigo? explico mas detalladamente?)

Para añadir un nuevo proyecto a Eclipse tengo que importarlo desde Maven, puedo hacerlo añadiendo un proyecto de Maven existente. Una vez seleccionado el proyecto, hay configurar la ejecución del proyecto. Debemos ir a "Run Configuration" y seleccionar el constructor de Maven. Una vez ahí, puedo añadirle un nombre específico y seleccionando el directorio base el proyecto a ejecutar y por último el "Goal", o meta debe ser "gwt:run", ya que en caso contrario no se producirá una ejecución correcta.

Antes de la ejecución se debe instalar el constructor de Maven desde el debuguer, sino puede dar algunos errores. Una vez finalizada la instalación, podremos ejecutar un proyecto con GWT apareciendo así el modo desarrollador y ahí se podrá lanzar la aplicación en el buscador por defecto.
\capitulo{6}{Trabajos relacionados}

En este apartado vamos a ver otros proyecto, estudios o trabajos que están relacionados con el proyecto aquí presentado.


Aquí tienes que nombrar las aplicaciones que existen y que serían: competidoras tuyas:
webthoth.herokuapp.com
https://github.com/izuzak/
noam
JSFLAP
http://hackingoff.com/compilers.
http://cgosorio.es/BURGRAM/
\capitulo{7}{Conclusiones y Líneas de trabajo futuras}

En este apartado, mencionaremos algunas conclusiones a las que hemos llegado después de realizar el proyecto además de posibles ideas relacionadas con futuros proyectos.

\section{Conclusiones}

Si bien es verdad que la dirección tomada en un principio sobre qué herramienta utilizar para desarrollar el proyecto, podría haber sido más acertada, hemos aprendido mucho sobre el entorno de GWT. \emph{WebSwing} nos hubiera facilitado mucho el trabajo de transformar a HTML5 el código en Java. Pero por otro lado, hemos visto la forma en la que se desarrollan cientos de aplicaciones con GWT, algunas tan grandes como Cloudorado\footnote{\url{https://www.cloudorado.com/}}, BookedIN\footnote{\url{https://bookedin.com/}}, Gae-Studio\footnote{\url{https://dev.arcbees.com/gaestudio/}} o Sigmah\footnote{\url{http://www.sigmah.org/EN.html}}.

En el caso de que hubiéramos hecho el proyecto con herramientas más fáciles nos hubieramos centrado sobre todo en añadir mejoras de Thoth, sin perder tanto tiempo en adaptar el código Java para GWT. 

En cuanto a mi, en líneas generales, he podido aprender mucho sobre el desarrollo web, la arquitectura que utiliza y como se distribuye el programa entre el cliente y el servidor o la forma en la que se despliega una aplicación de estas características.

\section{Líneas de trabajo futuras}

Como posibles mejoras sobre este proyecto, hemos encontrado varios detalles que se nos han escapado, ya sea por falta de tiempo, o por no habernos dado cuenta antes.
\begin{itemize}
\item Creemos que es interesante almacenar en la base de datos la configuración del usuario, para que una vez que cambie el idioma de la aplicación, por defecto, al iniciar sesión lo haga con el idioma elegido anteriormente.

\item Añadir algún método para guardar y cargar gramáticas, ya sea de forma local, en ordenador del usuario, o en la base de datos.

\item Añadir comentarios al código, de forma que no se tengan en cuenta a la hora de comprobar la gramática. Nos dimos cuenta tarde, al hacer varias pruebas finales.
\item Este proyecto se podría haber hecho de forma parecida pero con \emph{WebSwing} y centrándose sobre todo en la base de datos, en la accesibilidad desde distintos dispositivos etc.

\item Crear perfiles de usuario, en que pueda configurar un avatar, añadir gramáticas favoritas entre otros.

\item Añadir un panel de construcción de autómatas como en otras versiones de Thoth.
\end{itemize}


\bibliographystyle{plain}
\bibliography{bibliografia}

\end{document}
