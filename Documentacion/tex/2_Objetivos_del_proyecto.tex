\capitulo{2}{Objetivos del proyecto}

El objetivo principal de este trabajo de final de grado consiste en transformar el proyecto previo de Thoth, que está escrito en Java, a la tecnología web JavaScript. Esta conversión se llevará a cabo por medio de GWT, herramienta elegida posteriormente a un estudio previo (disponible en el capítulo Técnicas y Herramientas). 

El desarrollo consiste en una página web dinámica que sea capaz de hacer las mismas funcionalidades sobre gramáticas formales y sus algoritmos que en Thoth, poniendo a prueba mis propios conocimientos sobre Java así como otros lenguajes sobre tecnologías web como son HTML, CSS o JavaScript.

Pero la razón por la que pretendemos pasar de una aplicación de escritorio a otra que se ejecuta en el navegador es básicamente porque no es necesario instalar o descargar material para poderla utilizar Thoth, simplemente algunos conocimientos básicos sobre gramáticas formales. Por lo tanto aporta una clara ventaja con respecto a aplicaciones denominadas <<de escritorio>> como es el Thoth original sobre el que nos basamos en este proyecto.

La conversión de Thoth a GWT es necesaria porque amplía las posibilidades de mejora de las anteriores versiones. Pretendemos aportar mayor funcionalidad debido a que, al ser una aplicación web, podemos hacer un registro de los usuarios, con inicio de sesión, registro de actividades, etc. 

De hecho, uno de los puntos fuertes de este proyecto consiste en aprovechar la mayor parte de utilidad de las aplicaciones web. Aunque en las anteriores versiones la pretensión era sobre todo didáctica en materia relacionada con el estudio de los procesos del lenguaje, ahora, además de eso, también queremos llegar a poder saber la utilidad que se le da a Thoth. Para ello, gracias al registro de información sobre los usuarios, podremos saber cuando se ha registrado alguien, o iniciado sesión, qué gramáticas ha usado y más funcionalidades que puedan surgir. Eso si, siempre manteniendo la privacidad del usuario, cifrando la contraseña de registro.

Por último está el objetivo de experimentar con herramientas que no conocemos, y de esta forma ir aprendiendo a solucionar problemas nuevos. Al fin y al cabo lo que se trata de aprender en la universidad es superar estos retos por uno mismo, aplicando las técnicas que se nos enseñan.

Los objetivos del proyecto quedan reflejados de esta manera:

\begin{itemize}

\item Estudio de aplicaciones de conversión automática de Java a la web.
\item Convertir la parte de las gramáticas de Thoth mediante GWT
\item Crear un sistema de registro de usuarios para el acceso a la aplicación.
\item Crear un sistema de sesiones de acuerdo al registro.

\end{itemize}
