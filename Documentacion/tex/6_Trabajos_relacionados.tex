\capitulo{6}{Trabajos relacionados}

En este apartado vamos a ver otros proyectos, estudios o trabajos que están relacionados con este trabajo. Explicaremos en qué consisten y qué tienen que ver con este proyecto.

\begin{itemize}

\item \textbf{Noam}: Es una biblioteca JavaScript para trabajar con autómatas y gramáticas formales para lenguajes regulares y sin contexto.

\item \textbf{JSFLAP}\footnote{\url{http://jsflap.com/}}: Se trata de una aplicación web en desarrollo, programada en JavaScript, que sirve para visualizar lenguajes formales y teoría de los autómatas. Actualmente cuenta con una versión beta, aunque su desarrollo parece que se encuentra parado. Es un programa muy visual, permite exportar un autómata en \LaTeX, en texto o en imagen o configurar el aspecto visual. Es muy útil y fácil de usar.


\item \emph{\textbf{Compiler Construction Toolkit}}\footnote{\url{http://hackingoff.com/compilers}}: Se trata de una herramienta web con la que podemos construir componentes de un compilador. Sus herramientas se pueden clasificar en
\begin{itemize}
\item Herramientas de teoría de compiladores.
\item Herramientas de diseño de compiladores.
\item Herramientas para generar un parser.
\end{itemize}
Es algo más difícil de utilizar que JSFLAP, pero cuenta con un mayor número de herramientas. Genera código en lenguaje <<Ruby>>.

\item \emph{BURGRAM}\footnote{\url{http://cgosorio.es/BURGRAM/}}: Consiste en un programa para la generación y simulación de tablas de análisis sintáctico, Dirigido por el Dr. César Ignacio García Osorio y desarrollado por Carlos Gómez Palacios.
\end{itemize}