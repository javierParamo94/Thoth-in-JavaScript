\capitulo{1}{Introducción}

Este proyecto nació con el objetivo de llevar Thoth \cite{garcia2007ensenanza}, un antiguo proyecto escrito en Java, a la web. Para ello se utilizarán tecnologías web, que puedan ser utilizadas en diferentes dispositivos haciéndolo accesible a todo el mundo. Con ayuda de la herramienta conocida como GWT o \emph{Google Web Toolkit} \footnote{\url{http://www.gwtproject.org/}} según su denominación inglesa, traduciré la aplicación a Javascript en el lado del cliente haciendo posible la utilización de la aplicación directamente desde un navegador, algo que antes era imposible.

Thoth \cite{garcia2007ensenanza}
 es un antiguo proyecto enfocado a la actividad docente y relacionado con los procesadores de lenguaje, que fue realizado por varios alumnos de la Universidad de Burgos como trabajo de fin de carrera. Esta aplicación cuenta con dos versiones  hasta la fecha y con otros desarrollos como Web Thoth \cite{jute2017}.

Una de las principales preguntas que nos podemos hacer al ver este proyecto es el por qué traducir la aplicación al lenguaje JavaScript. 
En primer lugar todos los navegadores actuales son capaces de interpretar el código escrito en JavaScript y soportan al menos alguna de las versiones ECMAScript \cite{ecma:versiones}, siendo la última la séptima edición disponible desde 2016. Con ayuda de la tecnología AJAX, se puede ejecutar la aplicación en el cliente, es decir, en el navegador de un usuario mientras se mantiene la comunicación asíncrona con el servidor en segundo plano.

Utilizo GWT para este proyecto como herramienta para transformar la aplicación de escritorio a entornos web. Pese a que actualmente no es el mejor \emph{framework} para desarrollar, sí es una de las base de otro mucho más moderno y potente. Hablo de Vaadin, que se encuentra más actualizado y con más bibliotecas para un aspecto visual más moderno, pero que para poder hacer disfrutar de su versión completa, se necesita una licencia bajo pago. La forma de trabajar con GWT es creando el código en Java y el compilador hará una traducción a los lenguajes JavaScript y HTML.

Una de las cuestiones más importantes de este \emph{frameworks} es que GWT no es capaz de admitir todas las bibliotecas de Java y por eso debimos reescribir el código adaptándolo a las capacidades de este \emph{framework}. Encontramos otras opciones de las que posteriormente hablaré, que son capaces de hacer la traducción completa de Thoth en Java a HTML5 de una forma muchísimo más sencilla y directa sin necesidad, si quiera, de esforzarse demasiado. Pero ese no es el objetivo de un trabajo de fin de grado, sino el de plantear un desafío suficientemente grande con el que el alumno pueda hacer uso del conjunto de muchos de los conocimientos aprendidos a lo largo de grado.