\capitulo{3}{Conceptos teóricos}

En mi experiencia como profesor de programación, diseño 3D y robótica para alumnos de primaria, lo primero que les enseñábamos era a definir la programación como el lenguaje de comunicación entre nosotros, los humanos, y los ordenadores o los robots. Todo esto pertenece a los conceptos más puramente teóricos sobre la computación y que es, en sí, la base de la informática. 

Pues bien, a <<grosso modo>> el concepto es el mismo ahora. Como cada uno, hombre y máquina, <<habla>> un idioma diferente se debe establecer un lenguaje que sea la vía de comunicación entre ambos. 

Para poder entender todo lo que se va a tratar a continuación, debo explicar primero el concepto de \textit{autómata} o \textit{máquina abstracta}. Un autómata es un modelo matemático o un dispositivo teórico que recibe una cadena de símbolos como entrada y que al procesarla, genera un cambio de estado produciendo una salida determinada. Esta salida pude reconocer palabras y determinar si la entrada pertenece a un determinado lenguaje o no. En el símil anterior digamos que es el corrector, que cuando escribe el humano algo determina si esta bien escrito o no porque no lo va a poder entender el ordenador. 

Entonces ya sabemos que para que haya comunicación entre un ordenador o robot y un humano, tiene que haber un lenguaje y un autómata. Pero también algo más: una gramática.

\section{¿ Qué es una gramática y para qué sirve?}

Una gramática formal es un mecanismo para la generación de cadenas de caracteres que son admitidas por un determinado lenguaje formal, y que utiliza un conjunto de reglas de formación. Por lo tanto, podemos entenderlo dentro del concepto de las ciencias de la computación y la lógica matemática. Las cadenas de caracteres resultantes son a su vez <<bien formadas>> cuando pertenecen al lenguaje formal con el que se trabaja.\cite{aho1986compilers}

¿ Y porque es tan importante? Siguiendo con el ejemplo del principio, la gramática es lo que va a determinar si lo que se introduce en el autómata es correcto o no. Un conjunto de reglas que nos indicará el porque al juntar una serie de caracteres de una forma se van a poder entender.

Por otro lado, la denominación de la gramática formal desde un punto de vista más concreto, de denomina como una cuádrupla compuesta por:

\begin{itemize}
	\item Un alfabeto de símbolos terminales o tókenes denominado con la letra griega $\Sigma$.
	\item $\mathcal{N}$ que es un alfabeto formado por símbolos no terminales
	\item Un alfabeto de producciones denominado $\mathcal{P}$.
	\item Y por último un símbolo llamado axioma o símbolo inicial el cual $\mathcal{S} \in \mathcal{N}$.
\end{itemize}

El alfabeto total que compone la gramática esta formado, según lo anterior, por $\Sigma\cup\mathcal{N}$ , es decir, por el conjunto de los símbolos terminales y no terminales.

Pongamos un ejemplo de una gramática simple y veamos de que esta formada. Se suele utilizar el sistema de notación  \[x \rightarrow y\] \[z \rightarrow w\] para indicar una o varias producciones, en vez de \[(x, y) \in \mathcal{P} \] \[(z, w) \in \mathcal{P} \] siendo $\mathcal{P}$ el conjunto de producciones.

Por otro lado si hay más de una producción que comience con el mismo elemento la notación sería de esta forma \[ x \rightarrow y | z | w\] en lugar de ser \[ x \rightarrow y, x \rightarrow z, x \rightarrow w\]


Producción: 

En aquellos proyectos que necesiten para su comprensión y desarrollo de unos conceptos teóricos de una determinada materia o de un determinado dominio de conocimiento, debe existir un apartado que sintetice dichos conceptos.

Algunos conceptos teóricos de \LaTeX \footnote{Créditos a los proyectos de Álvaro López Cantero: Configurador de Presupuestos y Roberto Izquierdo Amo: PLQuiz}.

\section{Cifrado}

Las secciones se incluyen con el comando section.

\subsection{Subsecciones}

Además de secciones tenemos subsecciones.

\subsubsection{Subsubsecciones}

Y subsecciones. 


\section{Referencias}

Las referencias se incluyen en el texto usando cite \cite{wiki:latex}. Para citar webs, artículos o libros \cite{koza92}.


\section{Imágenes}

Se pueden incluir imágenes con los comandos standard de \LaTeX, pero esta plantilla dispone de comandos propios como por ejemplo el siguiente:

\imagen{escudoInfor}{Autómata para una expresión vacía}



\section{Listas de items}

Existen tres posibilidades:

\begin{itemize}
	\item primer item.
	\item segundo item.
\end{itemize}

\begin{enumerate}
	\item primer item.
	\item segundo item.
\end{enumerate}

\begin{description}
	\item[Primer item] más información sobre el primer item.
	\item[Segundo item] más información sobre el segundo item.
\end{description}
	
\begin{itemize}
\item 
\end{itemize}

\section{Tablas}

Igualmente se pueden usar los comandos específicos de \LaTeX o bien usar alguno de los comandos de la plantilla.

\tablaSmall{Herramientas y tecnologías utilizadas en cada parte del proyecto}{l c c c c}{herramientasportipodeuso}
{ \multicolumn{1}{l}{Herramientas} & App AngularJS & API REST & BD & Memoria \\}{ 
HTML5 & X & & &\\
CSS3 & X & & &\\
BOOTSTRAP & X & & &\\
JavaScript & X & & &\\
AngularJS & X & & &\\
Bower & X & & &\\
PHP & & X & &\\
Karma + Jasmine & X & & &\\
Slim framework & & X & &\\
Idiorm & & X & &\\
Composer & & X & &\\
JSON & X & X & &\\
PhpStorm & X & X & &\\
MySQL & & & X &\\
PhpMyAdmin & & & X &\\
Git + BitBucket & X & X & X & X\\
Mik\TeX{} & & & & X\\
\TeX{}Maker & & & & X\\
Astah & & & & X\\
Balsamiq Mockups & X & & &\\
VersionOne & X & X & X & X\\
} 
