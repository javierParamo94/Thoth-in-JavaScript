\capitulo{3}{Conceptos teóricos}

Como conceptos teóricos, me veo en la obligación de dar, primero, una definición general de lo que es una <<gramática formal>> antes de pasara a otras cuestiones que, por supuesto, tendrán que ver con esta.

Una gramática formal es un mecanismo para la generación de cadenas de caracteres que son admitidas por un determinado lenguaje formal, y que utiliza un conjunto de reglas de formación. Por lo tanto, podemos entenderlo dentro del concepto de las ciencias de la computación y la lógica matemática. Las cadenas de caracteres resultantes son a su vez <<bien formadas>> cuando pertenecen al lenguaje formal con el que se trabaja.

Por otro lado, la denominación de la gramática formal desde un punto de vista más concreto, de denomina como una cuádrupla compuesta por:

\begin{itemize}
	\item Un alfabeto de símbolos terminales o tokenes denominado con la letra griega $\Sigma$ .
	\item  $N$ que es un alfabeto formado por símbolos no terminales
	\item Un alfabeto de producciones denominado $P$
	\item Y por último un símbolo llamado axioma o símbolo inicial el cual $S \in N$
\end{itemize}

El alfabeto total que compone la gramática esta formado, según lo anterior, por $\Sigma \cup N$ , es decir, por el conjunto de los símbolos terminales y no terminales.

Pongamos un ejemplo de una gramática simple y veamos de que esta formada. Se suele utilizar el sistema de notación  \[x \rightarrow y\] \[z \rightarrow w\] para indicar una o varias producciones, en vez de \[(x, y) \in P \] \[(z, w) \in P \] siendo $P$ el conjunto de producciones.

Por otro lado si hay más de una producción que comience con el mismo elemento la notación sería de esta forma \[ x \rightarrow y | z | w\] en lugar de ser \[ x \rightarrow y, x \rightarrow z, x \rightarrow w\]


Producción: 

En aquellos proyectos que necesiten para su comprensión y desarrollo de unos conceptos teóricos de una determinada materia o de un determinado dominio de conocimiento, debe existir un apartado que sintetice dichos conceptos.

Algunos conceptos teóricos de \LaTeX \footnote{Créditos a los proyectos de Álvaro López Cantero: Configurador de Presupuestos y Roberto Izquierdo Amo: PLQuiz}.

\section{Secciones}

Las secciones se incluyen con el comando section.

\subsection{Subsecciones}

Además de secciones tenemos subsecciones.

\subsubsection{Subsubsecciones}

Y subsecciones. 


\section{Referencias}

Las referencias se incluyen en el texto usando cite \cite{wiki:latex}. Para citar webs, artículos o libros \cite{koza92}.


\section{Imágenes}

Se pueden incluir imágenes con los comandos standard de \LaTeX, pero esta plantilla dispone de comandos propios como por ejemplo el siguiente:

\imagen{escudoInfor}{Autómata para una expresión vacía}



\section{Listas de items}

Existen tres posibilidades:

\begin{itemize}
	\item primer item.
	\item segundo item.
\end{itemize}

\begin{enumerate}
	\item primer item.
	\item segundo item.
\end{enumerate}

\begin{description}
	\item[Primer item] más información sobre el primer item.
	\item[Segundo item] más información sobre el segundo item.
\end{description}
	
\begin{itemize}
\item 
\end{itemize}

\section{Tablas}

Igualmente se pueden usar los comandos específicos de \LaTeX o bien usar alguno de los comandos de la plantilla.

\tablaSmall{Herramientas y tecnologías utilizadas en cada parte del proyecto}{l c c c c}{herramientasportipodeuso}
{ \multicolumn{1}{l}{Herramientas} & App AngularJS & API REST & BD & Memoria \\}{ 
HTML5 & X & & &\\
CSS3 & X & & &\\
BOOTSTRAP & X & & &\\
JavaScript & X & & &\\
AngularJS & X & & &\\
Bower & X & & &\\
PHP & & X & &\\
Karma + Jasmine & X & & &\\
Slim framework & & X & &\\
Idiorm & & X & &\\
Composer & & X & &\\
JSON & X & X & &\\
PhpStorm & X & X & &\\
MySQL & & & X &\\
PhpMyAdmin & & & X &\\
Git + BitBucket & X & X & X & X\\
Mik\TeX{} & & & & X\\
\TeX{}Maker & & & & X\\
Astah & & & & X\\
Balsamiq Mockups & X & & &\\
VersionOne & X & X & X & X\\
} 
