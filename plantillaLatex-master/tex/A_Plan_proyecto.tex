\apendice{Plan de Proyecto Software}

\section{Introducción}

\section{Planificación temporal}

\subsection{Sprint 0 (1/2/2017 - 8/2/2017)}

En este primer <<Sprint>> determinamos que posibles herramientas de traducción pueden ser útiles para este proyecto sin profundizar demasiado. Se hace también la primera toma de contacto con Thoth para ver su funcionamiento básico.

Por otro lado, analizamos las posibilidades del proyecto, determinando los caminos en los que puede derivar dicho proyecto. Es decir, en estos primeros pasos no sabemos como funcionan estas herramientas, es por ello que el resultado sea más simple del esperado o por el contrario resulten complicaciones que lleven un tiempo mayor al esperado.

Además de esto, hacemos una introducción a las herramientas de documentación y gestión. Por ello aclaramos que:

\begin{itemize}
\item Para la gestión de versiones haremos uso de GitHub, asociando un gestor de tareas llamado Zenhub que funciona como un <<plugin>> en el buscador.
\item Realizo un primer contacto con \LaTeX 
\end{itemize}

En esta primeras semanas no me aclaro mucho con el funcionamiento del gestor de versiones, y es por ello que no hago un buen uso de los <<commits>> ni de los <<issues>> que proporciona Github.

\subsection{Sprint 1 (8/2/2017 - 15/2/2017)}

Ya en la segunda semana realizo una evaluación más exhaustiva de las herramientas de traducción, analizando los pros y los contras de ellas. Por lo tanto tomamos la decisión de centrarnos en GWT como la principal y con la que vamos a llevar a cabo el proyecto.

Definimos como tareas semanales:

\begin{itemize}
\item Evaluar los pros y contras de las diferentes herramientas de traducción de código.
\item Documentar esa evaluación con \LaTeX , familiarizándome con la manera de documentar.
\item Realizar las primeras pruebas con GWT, de una forma simple.
\end{itemize}

En esta semana hago alguna prueba simple con GWT, gracias a los ejemplos que proporciona la página oficial a modo de tutorial. También realizo alguna prueba simple con JSweet para ver su funcionamiento real y si es, de verdad, útil para poder llevar a cabo el proyecto. En el último ejemplo que hago me ocurre un problema con GWT que no termino de solucionar y que me obliga a posponer la prueba de ese ejemplo para el siguiente sprint.

También profundizo algo más en la documentación y en el uso de las herramientas para documentar. Hasta este punto no he asociado las tareas o <<issues>> con los <<milestones>> y por lo tanto no queda registrado el tiempo del sprint.

\subsection{Sprint 2 (15/2/2017 - 22/2/2017)}

En este tercer sprint, lo primero que hago es solucionar el anterior problema que tuve con GWT. Consistía en configurar bien el entorno de Eclipse y el <<plugin>> para poder hacer la ejecución de GWT con <<Super Dev Mode>>, alternativa implantada por los desarrolladores para evitar la necesidad de instalar una extension de GWT en el buscador en el cual se lanza la aplicación.

También en esta semana descubrimos una nueva herramienta relacionada con el tema, que se llama Vaadin y que nos puede servir, por lo menos para hacer una comparativa más completa de las herramientas de traducción. 

La parte más importante de esta semana es la prueba de traducción del <<core>> de Thoth, que aunque no sale como esperamos, ha resultado útil para conocer con mayor profundidad tanto la aplicación como la herramienta.

Por lo tanto como tareas para este sprint:

\begin{itemize}
\item Solucionar el error surgido con GWT.
\item Realizar pruebas de traducción con el <<core>> de Thoth.
\item Incluir la nueva herramienta en la comparativa.
\end{itemize}

Muy a mi pesar, en el <<milestone>> de esta semana, aunque he pasado cada tarea al estado de realizada o <<done>> no las he cerrado hasta darnos cuenta al final del sprint, es por ello que el <<burndown>> queda de esta manera.


\subsection{Sprint 3 (22/2/2017 - 1/3/2017)}

Ya en la cuarta semana se hace un intento más completo para traducir la aplicación. Como pudimos comprobar en la semana pasada, GWT no traducía las librerías de la parte visual de Thoth. Es por ello decidimos hacer un ejemplo de forma manual que consiste en programar parte de la vista que esta asociada a las partes más relevantes del núcleo

Por ejemplo, nos centramos en hacer una prueba con la gramática, que esta dentro del núcleo, creando una pantalla con un <<text label>> para comprobar si funcionaba la traducción de esa parte del núcleo. De esta forma podríamos ver como hacía la traducción de todo el núcleo, ya que las otras partes de las que se compone son similares en el uso de librerías y bibliotecas.

Quedan así asignadas las tareas para del sprint número tres:

\begin{itemize}
\item Transformar la Gramática del núcleo de Thoth.
\item Transformar el Autómata del núcleo.
\item Transformar la simulación del núcleo de Thoth.
\item Documentar toda esta parte.
\end{itemize}

Al final solo se pudo llevar a cabo la tares de la Gramática y la documentación porque no se pudo avanzar a las demás. Nuestra idea era probar a tratar de traducir todo, núcleo incluido, ejecutando esas partes en el cliente de GWT  pero vimos que esto no es posible. Intentamos hacerlo de varias formas eliminando partes no esenciales de la aplicación para reducir errores de compilación hasta darnos por vencidos y ver que esa no era la solución.

\subsection{Sprint 4 (1/3/2017 - 8/3/2017)}
La cuarta es la semana en la cual, hemos intentado pasar la aplicación en la parte del servidor y probar por nuestra cuenta. Sólo hemos metido el núcleo en el paquete servidor para ver que surgía. Como vimos que no había una comunicación entre el cliente y el servidor hicimos varios intentos, probando con el paquete de <<shared>> o compartido, pero GWT también traduce ese paquete a JavaScript por lo que seguía dando los mismo errores que en el cliente.

Por lo tanto en este sprint tenemos esta tareas:

\begin{itemize}
\item Solucionar un error en la traducción.
\item Realizar pruebas cliente-servidor con el núcleo de la aplicación.
\item Cambios y mejoras en la documentación.
\end{itemize}

Una vez nos dimos cuenta de que el fallo de tratar de hacer la aplicación en la parte del servidor era que GWT no reconocía algunas de las librerías claves, tanto en la parte visual como en el núcleo de la aplicación, decidimos buscar otros caminos alternativos. 

El funcionamiento de GWT consiste en traducir a <<JavaScript>> la parte del cliente y la compartida. En consecuencia decidimos hacer pruebas en las que las partes mas fundamentales del núcleo se encontrasen en el lado del servidor. De esta forma cuando el cliente necesitase hacer algún uso de métodos con librerías no reconocidas por GWT, simplemente llamase al servidor ya que este podría soportar dichos métodos. 

En los primeros intentos nos dimos cuenta de que estas llamadas no se podían hacer de una forma simple, ya que la comunicación entre cliente y servidor no funcionaba y no obteníamos los resultados que esperábamos. Aún así seguimos haciendo pruebas para asegurarnos, metiendo dentro del paquete <<compartido>> las partes del núcleo mas cercanas a lo que nosotros consideramos la vista. El problema seguía siendo esa comunicación. Interpretaba como del lado del cliente lo que nosotros queríamos que formara parte del servidor, dando errores debido a que GWT no trabaja con esas librerías.

\subsection{Sprint 5 (8/3/2017 - 15/3/2017)}

Principalmente, en este quinto sprint, se llevan a cabo las pruebas para entender y poder evaluar la comunicación cliente-servidor, por medio de unos ejemplos. Además de eso, planteamos la idea de realizar un <<login>> y validación de usuarios, pero solo como idea, ya que no es de gran importancia.


Así que en este sprint tenemos esta tareas:

\begin{itemize}
\item Ejemplo cliente-servidor con GWT.
\item Login y validación en GWT.
\end{itemize}

La comunicación entre el cliente y el servidor se lleva a cabo mediante la comunicación RPC (Remote Procedure Call). Es por ello que se hace necesario entender y practicar el funcionamiento de esta práctica. Los ejemplos realizados han sido dos: el primero es un ejemplo o tutorial ofrecido por la página oficial de GWT, que consiste en hacer un visor del <<stock>> que cambia de forma aleatoria sus valores. Y el segundo ejemplo consistió en hacer un pequeño ejemplo de llamada de funciones con más clases que en el anterior.


\subsection{Sprint 6 (15/3/2017 - 22/3/2017)}

En esta semana nos metemos ya en serio con la aplicación propiamente dicha. Lo primero que hacemos es conseguir que la comunicación entre en núcleo (ya hecho) y su uso sea fluido. Para ello lo que hacemos es incluir algunas partes en el cliente y otras en el servidor. En el cliente sólo podemos incluir las clases más simples, más primitivas de la aplicación porque su contenido es entendido por GWT y puede hacer la traducción a JavaScript sin problemas de librerías.

Por lo tanto las tareas son básicamente dos:

\begin{itemize}
\item Llevar a cabo el primer prototipo o sección de la aplicación.
\item Documentar y corregir errores anteriores en la documentación.
\end{itemize}

La realización del prototipo nos lleva tiempo ya que se necesita comprender muy bien el funcionamiento interno de Thoth y así poder definir un diseño del software adecuado según ese funcionamiento. Una vez hecho eso parece simplificarse los problemas que al principio se tenían.

\subsection{Sprint 7 (22/3/2017 - 29/3/2017)}

Parece ser que en esta octava semana ya podemos empezar a trabajar más en la programación del diseño e implementar funcionalidades. Lo primero que tenemos que hacer es que esa comunicación entre métodos de resultados <<más>> visibles e integrarlos en una <<GUI>> denominada en español como interfaz gráfica de usuario. 

Así es que definimos como tareas las siguientes:

\begin{itemize}
\item Reestructurar la aplicación y limpiar el código.
\item Mejorar la GUI con Vaadin, ya que la anterior es muy básica.
\item Implementar el algoritmo "Eliminar símbolos no terminales"
\end{itemize}

Primeramente hay que organizar el código haciéndolo más legible y limpiarlo de comentarios, y pruebas para ver su funcionamiento. Queremos que los resultados queden de una forma similar al Thoth original, para conservar su esencia, usabilidad, y buen diseño. Para ello hacemos uso de Vaadin, una herramienta que se puede utilizar como un <<plugin>> en eclipse y que se integra perfectamente con GWT.

\subsection{Sprint 8 (29/3/2017 - 5/4/2017)}

Al principio de este octavo sprint o 9 semana, estuve pendiente de la respuesta por parte de Vaadin sobre si me podía conceder o no una licencia gratuita, así que me centré en incluir el algoritmo de eliminación de símbolos no terminales, mejorando lo que tenía hasta el momento, que era solo una pequeña interfaz con una funcionalidad que no era la exacta. Por ello me centré en corregirla. Así lo hicimos.
Posteriormente llegó la respuesta de Vaadin explicando que no me podía conceder la licencia y que buscase otras opciones dentro de las que ellos mismos me ofrecían. Al principió lo intenté pero resulto que no supe como hacerlo. Lograba hacer un proyecto de Vaadin pero no conseguía incluirlo en el mio propio de GWT.

Así que comencé a hacer la interfaz por mi mismo, con las posibilidades de GWT.

Las tareas para esa semana fueron: 

\begin{itemize}
\item La inclusión del algoritmo de Eliminacion de símbolos no terminales, de una forma mejorada.
\item Recabar información sobre internacionalización en GWT.
\end{itemize}

Logramos hacer que funcionase el algoritmo como queríamos y estuvimos mejorando el diseño y la funcionalidad de la interfaz de dicho algoritmo.

\subsection{Sprint 9 (05/4/2017 - 19/4/2017)}

Es el sprint más largo hasta la fecha ya que incluye dos semanas de trabajo por coincidir con las vacaciones de Semana Santa. Engloba fundamentalmente el hacer los demás algoritmos, con sus respectivas vistas.

\begin{itemize}
\item Implementar los algoritmos restantes.
\item Mejorar el código, haciéndolo más comprensible y organizar la parte visual.
\item Hacer la documentación sobre la parte del diseño y sobre las herramientas que tiene la UBU relacionadas con este proyecto.
\end{itemize}

En estas dos semanas no pude realizar todos los algoritmos como en un principio pretendía porque la parte visual se alejaba de lo que tenia hecho hasta el momento, es decir, necesitaba hacer nuevas vistas que llevaban más tiempo del pensado y no me dio tiempo, pero si que incluí la mayor parte de ellos. Estuvimos estudiando como aplicar el resaltado de las producciones. En el Thoth original utilizaba una librería, java swing, que ya comprobamos anteriormente que no podía ser incluida en GWT en este proyecto así que buscamos otras alternativas como hacer un resaltado a mano con HTML. De momento lo dejamos ahí.

\subsection{Sprint 10 (19/4/2017 - 26/4/2017)}

Una vez pasadas las vacaciones de Semana Santa, el proyecto tiene una forma más madura y podemos ir haciendo añadidos más funcionales a la aplicación. Es por ello que decidimos empezar con la internacionalización, viendo como funcionaba para poder entenderla y una vez entendida incluirla en el proyecto. Además de esto, después de comprobar en la semana pasada que el resaltado lo podíamos hacer con HTML nos dedicamos de lleno a ello. La verdad es que nos llevó muchos quebraderos de cabeza. El porqué no era otra cosa que teníamos que tener en cuenta todo el funcionamiento interno, <<destripar>> que contenía cada variable, como funcionaban cada método de GWT etc. Todo este trabajo nos llevo mucho tiempo, ya que estuvimos todo el sprint a base de prueba y error con los diferentes algoritmos hasta que funcionó en todos de la forma que deseamos.

Estas fueron las tareas correspondientes a esta semana.

\begin{itemize}
\item Aplicar internacionalización.
\item Cambiar la visualización de los paneles.
\item Resaltado de producciones en los algoritmos con HTML.
\item Incluir los demás algoritmos.
\item Incluir pestañas y elementos de Vaadin.
\end{itemize}

La parte visual, es decir, la inclusión de pestañas en realidad no lo llegamos a aplicar bien, de la forma deseada y se trató en el siguiente sprint. Además los elementos de Vaadin no se pudieron incluir ya que la única forma de añadirlos a un proyecto GWT son con licencias de pago. La visualización de los paneles simplemente fue una recolocación para que quedase más agradable a la vista.


\subsection{Sprint 11 (26/4/2017 - 03/5/2017)}

En la semana 13, seguimos acumulando un pequeño <<bug>> en el resaltado de las producciones, y es que la interpretación de las comillas dobles <<``''>> no las interpretaba como nosotros queríamos y por ello no hacía un buen borrado del resaltado, manteniéndose en cada paso. La solución fue simplemente sustituirlas por las comillas dobles de java <</``>>. Pero las dos grandes tareas de esta semana consistieron en aplicar el algoritmos FirstFollow y un TabLayoutPanel para hacer un panel con pestañas como los que se ven en los navegadores web.

Las tareas quedaron asi.

\begin{itemize}
\item Solucionar bug en el resaltado.
\item Implementar algoritmo FirstFollow.
\item Incluir TabLayoutPanel para hacer diferentes pestañas.
\end{itemize}

El algoritmo FirstFollow fue fácil hasta el momento de mostrar los resultados en las tablas. Resulta que al tratar de imprimir los resultados de tipo Object pasándolos como un <<string>> no los reconocía bien. Este tipo de errores no son fáciles de detectar en GWT, ya que la forma para verlos claramente es imprimiendo los valores por pantalla y tratando de analizarlos, en que formato están etc. Al final descubrimos que el problema era al tratar de pasara al formato string valores que GWT interpretaba como <<undefined>> y que no eran más que valores en blanco, espacios en blanco dentro de la tabla. 

El otro gran quebradero de cabeza fue el tratar de incluir pestañas en la aplicación. No es nada fácil añadir una pestaña con una nueva vista sin que esta remplace otra, se superponga o cualquier otra cosa. A día de hoy no sabemos bien el porqué de esto al cien por cien. Lo que si es cierto es que no es completamente necesaria y se puede hacer un reemplazamiento de otras vistas.

\subsection{Sprint 12 (03/5/2017 - 10/5/2017)}
Entrado en el mes de mayo, la recta final del proyecto, y con una aplicación base ya consolidada, por denominarlo así, no dedicamos sobre todo ha hacer mejoras en cuanto a temas visuales y funcionales. Además de esto el objetivo en este mes es hacer un registro e inicio de sesión de los usuarios para así tener un control detallado del uso.
Con todo esto por un lado, también era hora de dedicarle un porcentaje de tiempo mayo a documentar el proyecto. 

Para esta semana determinamos que las tareas quedaría de esta manera:

\begin{itemize}
\item Comenzar a estudiar y probar el inicio de sesión.
\item Actualizar la documentación.
\item Mejoras en la interfaz (botones, colores y formas).
\item Añadir un control de errores más elaborado y hacer la aplicación más robusta.
\item Cambiar la funcionalidad de los botones en los algoritmos.
\end{itemize}

La tarea de cambiar la funcionalidad de los botones en los algoritmos hace referencia a que regrese o vaya a una vista u otra dependiendo del botón pulsado o de las acciones asociadas a ello.
Las mejoras visuales son hechas a mano, como ya he comentado anteriormente, y es por ello que es un trabajo largo, en el que hay que estar compilando continuamente para ver los efectos resultantes, que no siempre son los esperados.
El inicio de sesión acabó en un ejemplo con varios errores y algo chapucero que sirvió para comprender el funcionamiento por lo que no se mejoró nada en la aplicación.

\section{Estudio de viabilidad}

\subsection{Viabilidad económica}

\subsection{Viabilidad legal}


