\apendice{Plan de Proyecto Software}

\section{Introducción}

\section{Planificación temporal}

\subsection{Sprint 0 (1/2/2017 - 8/2/2017)}

En este primer <<Sprint>> determinamos que posibles herramientas de traducción pueden ser útiles para este proyecto sin profundizar demasiado. Se hace también la primera toma de contacto con Thoth para ver su funcionamiento básico.

Por otro lado, analizamos las posibilidades del proyecto, determinando los caminos en los que puede derivar dicho proyecto. Es decir, en estos primeros pasos no sabemos como funcionan estas herramientas, es por ello que el resultado sea más simple del esperado o por el contrario resulten complicaciones que lleven un tiempo mayor al esperado.

Además de esto, hacemos una introducción a las herramientas de documentación y gestión. Por ello aclaramos que:

\begin{itemize}
\item Para la gestión de versiones haremos uso de GitHub, asociando un gestor de tareas llamado Zenhub que funciona como un <<plugin>> en el buscador.
\item Realizo un primer contacto con \LaTeX 
\end{itemize}

En esta primeras semanas no me aclaro mucho con el funcionamiento del gestor de versiones, y es por ello que no hago un buen uso de los <<commits>> ni de los <<issues>> que proporciona Github.

\subsection{Sprint 1 (8/2/2017 - 15/2/2017)}

Ya en la segunda semana realizo una evaluación más exhaustiva de las herramientas de traducción, analizando los pros y los contras de ellas. Por lo tanto tomamos la decisión de centrarnos en GWT como la principal y con la que vamos a llevar a cabo el proyecto.

Definimos como tareas semanales:

\begin{itemize}
\item Evaluar los pros y contras de las diferentes herramientas de traducción de código.
\item Documentar esa evaluación con \LaTeX , familiarizándome con la manera de documentar.
\item Realizar las primeras pruebas con GWT, de una forma simple.
\end{itemize}

En esta semana hago alguna prueba simple con GWT, gracias a los ejemplos que proporciona la página oficial a modo de tutorial. También realizo alguna prueba simple con JSweet para ver su funcionamiento real y si es, de verdad, útil para poder llevar a cabo el proyecto. En el último ejemplo que hago me ocurre un problema con GWT que no termino de solucionar y que me obliga a posponer la prueba de ese ejemplo para el siguiente sprint.

También profundizo algo más en la documentación y en el uso de las herramientas para documentar. Hasta este punto no he asociado las tareas o <<issues>> con los <<milestones>> y por lo tanto no queda registrado el tiempo del sprint.

\subsection{Sprint 2 (15/2/2017 - 22/2/2017)}

En este tercer sprint, lo primero que hago es solucionar el anterior problema que tuve con GWT. Consistía en configurar bien el entorno de Eclipse y el <<plugin>> para poder hacer la ejecución de GWT con <<Super Dev Mode>>, alternativa implantada por los desarrolladores para evitar la necesidad de instalar una extension de GWT en el buscador en el cual se lanza la aplicación.

También en esta semana descubrimos una nueva herramienta relacionada con el tema, que se llama Vaadin y que nos puede servir, por lo menos para hacer una comparativa más completa de las herramientas de traducción. 

La parte más importante de esta semana es la prueba de traducción del <<core>> de Thoth, que aunque no sale como esperamos, ha resultado útil para conocer con mayor profundidad tanto la aplicación como la herramienta.

Por lo tanto como tareas para este sprint:

\begin{itemize}
\item Solucionar el error surgido con GWT.
\item Realizar pruebas de traducción con el <<core>> de Thoth.
\item Incluir la nueva herramienta en la comparativa.
\end{itemize}

Muy a mi pesar, en el <<milestone>> de esta semana, aunque he pasado cada tarea al estado de realizada o <<done>> no las he cerrado hasta darnos cuenta al final del sprint, es por ello que el <<burndown>> queda de esta manera.


\subsection{Sprint 2 (22/2/2017 - 1/3/2017)}

\section{Estudio de viabilidad}

\subsection{Viabilidad económica}

\subsection{Viabilidad legal}


