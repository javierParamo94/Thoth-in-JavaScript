\capitulo{2}{Objetivos del proyecto}

El objetivo principal de este trabajo de final de grado consiste en transformar el proyecto previo de Thoth, que esta escrito en Java, a la tecnología web JavaScript. Esta conversión se llevará a cabo por medio de GWT, herramienta elegida posteriormente a un estudio previo. 

Consiste pues, en desarrollar una página web dinámica que sea capaz de hacer las mismas funcionalidades sobre gramáticas formales y sus algoritmos que en Thoth, poniendo a prueba mis propios conocimientos sobre Java así como otros lenguajes sobre tecnologías web como son HTML, CSS o JavaScript.

La motivación principal es la tendencia actual de desarrollar este tipo de aplicaciones web, que resultan más accesibles a los usuarios ya que no necesitan requisitos adicionales a los de acceder, por ejemplo, a otros web como son Facebook o LinkeIn. No es necesario instalar o descargar material para poderla utilizar, simplemente algunos conocimientos básicos sobre gramáticas del lenguaje. Por lo tanto Aporta una clara ventaja con respecto a aplicaciones denominadas <<de escritorio>> como es el Thoth original sobre el que nos basamos en este proyecto.

Además pretendemos aportar alguna mejora con respecto al anterior proyecto en base a que al ser una aplicación web podemos hacer un registro de los usuarios, con inicio de sesión, registro de actividades, etc. 

De hecho uno de los puntos fuertes de este proyecto consiste en aprovechar la mayor parte de utilidad de las aplicaciones web. Aunque en las anteriores versiones la pretensión era sobre todo didáctica en materia relacionada con el estudio de los procesos del lenguaje, ahora a demás de eso también queremos llegar a poder saber la utilidad que se la da a Thoth gracias al ya mencionado registro de información sobre los usuarios, eso si, siempre manteniendo la privacidad del usuario. 

Para lograr además que el acceso sea sencillo y cómodo para el usuario final, pretendemos alojar el programa en un servidor, con lo que el rango de alcance sería como menos que de el mundo entero, independientemente del país o el idioma ya que cuenta con la propiedad de poder traducir los textos a algunos de los idiomas más comunes. El objetivo es siempre que el usuario se sienta cómodo usando la aplicación.