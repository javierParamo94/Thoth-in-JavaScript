\capitulo{2}{Objetivos del proyecto}

El objetivo principal de este trabajo de final de grado consiste en transformar el proyecto previo de Thoth, que esta escrito en Java, a la tecnología web denominada JavaScript. Esta conversión se llevará a cabo por medio de una herramienta automática, elegida posteriormente a un estudio previo. 

Consiste pues, en desarrollar una página web dinámica que sea capaz de hacer las mismas funcionalidades sobre gramáticas formales y sus algoritmos que en Thoth, poniendo a prueba mis propios conocimientos sobre Java así como otros lenguajes sobre tecnologías web como son HTML, CSS o JavaScript.

La motivación principal es la tendencia actual de desarrollar este tipo de aplicaciones web, que resultan más accesibles a los usuarios ya que no necesitan requisitos adicionales a los de acceder, por ejemplo, a otros web como son Facebook o Linkein. Aporta, por lo tanto, una clara ventaja con respecto a aplicaciones denominadas <<de escritorio>> como es el Thoth original sobre el que nos basamos en este proyecto.

Además pretendemos aportar alguna mejora con respecto al anterior proyecto en base a que al ser una aplicación web podemos hacer un registro de los usuarios, con inicio de sesión etc.