\capitulo{1}{Introducción}

Este proyecto nació con el objetivo de llevar Thoth, un antiguo proyecto escrito en Java, a la web usando para ello tecnologías adaptadas y que puedan ser utilizadas en diferentes dispositivos haciéndolo accesible a todo el mundo. Con ayuda de la herramienta conocida como GWT o Google Web Toolkit según su denominación inglesa, traduciré la aplicación a Javascript en el lado del cliente haciendo posible la utilización de la aplicación en la web, algo que antes era imposible.

Thoth \cite{garcia2007ensenanza}
 es un antiguo proyecto enfocado a la actividad docente y relacionado con los procesadores de lenguaje, que fue realizado por varios alumnos de la Universidad de Burgos como trabajo de fin de carrera. Esta aplicación cuenta con dos versiones  hasta la fecha y con otros desarrollos como Web Thoth.

Una de las principales preguntas que nos podemos hacer al ver este proyecto es el por qué traducir la aplicación al lenguaje JavaScript. 
En primer lugar todos los navegadores actuales son capaces de interpretar el código escrito en JavaScript y soportan al menos alguna de las versiones ECMAScript \cite{ecma:versiones}, siendo la última la séptima edición disponible desde 2016. Con ayuda de la tecnología AJAX, entre otras, este lenguaje es capaz de establecer comunicación cliente-servidor, algo que al principio de su desarrollo no lograba hacer. 

Utilizo GWT para este proyecto como herramienta para desarrollar la aplicación. Pese a que actualmente no es el mejor <<framework>> para desarrollar si es una de las bases de otros mucho más modernos y potentes. La forma de trabajar con GWT es creando el código en Java y el compilador hará una traducción a los lenguajes JavaScript y HTML.

La elección surgió como un desafío para nosotros ya que otras herramientas el esfuerzo era mínimo. Este desafío consiste en que GWT no es capaz de admitir todas las librerías de Java y por eso debimos reescribir el código adaptándolo a las capacidades de este <<framework>>. Si, encontramos otras opciones, de las que posteriormente hablaré, que son capaces de hacer la traducción completa de Thoth en Java a JavaScript de una forma muchísimo más sencilla y directa sin necesidad si quiera de esforzarse demasiado. Pero ese no es el objetivo de un trabajo de fin de grado, sino el de plantear un desafió suficientemente grande con el que el alumno pueda hacer uso del conjunto de muchos de los conocimientos aprendidos a lo largo de grado.