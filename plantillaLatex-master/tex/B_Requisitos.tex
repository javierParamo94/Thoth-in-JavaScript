\apendice{Especificación de Requisitos}

\section{Introducción}

En esta sección se definirán los requisitos de la aplicación así como los objetivos de una forma más esquemática y concreta que en otros apartados. De esta manera se puede ver en que dirección va orientado el proyecto y porque se ha elegido cada uno de acuerdo a los requisitos.

\section{Objetivos generales}

Estos has sido los principales objetivos del proyecto según la línea de tiempo.

\subsection{Crear un prototipo de aplicación cliente-servidor con GWT}

El objetivo de crearlo es el de poder ver el funcionamiento, comunicación y procesado del lenguaje Java en GWT para definir que limitaciones y posibilidades de un proyecto realizado con esta herramienta. Con ello se define que:  
\begin{itemize}
\item El lado del cliente y el compartido (shared) son los que se traducen al lenguaje JavaScript y los ficheros se crean dentro de un directorio <<war>>.
\item El lado del cliente tiene ciertas limitaciones y no acepta todo tipo de librerías de Java. 
\item La comunicación entre cliente y servidor es por medio de llamadas de procedimiento remoto o RPC.
\item Parte de la aplicación debe ir indudablemente en el servidor y comunicarse con el cliente para mostrar los resultados 
\end{itemize}

Este primer objetivo es fundamental y marca las posibilidades reales que ofrece GWT de cara a este proyecto.

\subsection{Construir la aplicación con la definición de gramática}

Consiste en hacer ya el comienzo de la aplicación empezando por la <<definición de gramática>> que es la parte de Thoth en la que se introduce un a gramática en un recuadro de texto y el programa es capaz de identificarla y mostrar sus propiedades. Para ello es necesario haber logrado en objetivo anteriormente descrito, ya que se necesita alojar en el servidor el <<parser>> de la gramática y hacerle peticiones desde el cliente para interpretarla y mostrar sus propiedades.

Es el comienzo de la aplicación y la base sobre la que trabajar posteriormente ya que para implementar los algoritmos es necesario tener definida una gramática.

\subsection{Construir los algoritmos sobre gramáticas}

El objetivo es implementar cada unos de los algoritmos que se aplican sobre una gramática en Thoth. Es una de las utilidades más importantes de esta aplicación y la cuál otorga mas posibilidades a los usuarios. Esta compuesta por los algoritmos siguientes:

\begin{itemize}
\item Eliminación de símbolos no terminales.
\item Eliminación de símbolos no alcanzables. 
\item Eliminación de símbolos anulables.
\item Eliminación de producciones no generativas.
\item Eliminación de recursividad directa.
\item Eliminación de recursividad indirecta.
\item Factorización por la izquierda.
\item Forma normal de Chomsky.
\item Análisis descendente LL(K) formado por el cálculo de <<First y Follow>> y reconocimiento por medio de TASP o Tabla de Análisis Sintáctico Predictivo.
\end{itemize}

Hay además dos funcionalidades que engloban varios de estos algoritmos como son la de limpiar gramática y eliminar recursividad.

Varois de ellos tienen la particularidad de que su ejecución esta hecha paso a paso, señalando dichos pasos con un resaltado con colores.

\subsection{Construir un registro de usuarios}

Es el objetivo puesto en último lugar debido a que la prioridad principal era que la aplicación funcionase correctamente y una vez conseguido añadir funcionalidades asociadas a ella. Consiste en, por medio de la aplicación cliente-servidor, hacer una base de datos con los usuarios que se conecten a la aplicación una vez que hayan sido registrados.

Se comprueba que los datos introducidos cumplen una serie de requisitos y se debe rellenar una información que se almacena en la base de datos.
El fin de este registro es para obtener información de quienes accedan a la aplicación y poder así analizar por ejemplo cuantos usuarios han hecho uso de ella.

\section{Catalogo de requisitos}

\section{Especificación de requisitos}


