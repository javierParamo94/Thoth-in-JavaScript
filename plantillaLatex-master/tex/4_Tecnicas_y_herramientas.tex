\capitulo{4}{Técnicas y herramientas}

Dentro de las diferentes herramientas que utilizaré para la realización de este trabajo, la más esencial es aquella con la cual, realizaré la conversión a JavaScript.

Es por ello esencial hacer una buena elección comparando y analizando la diferentes posibilidades a elegir.

Como principales herramientas para la conversión de Java a JavaScript he podido encontrar GWT (Google Web Toolkit), JSweet y DukeScript aunque también hay otras de las que hablaré más adelante. Comenzaré por GWT.

Google Web Toolkit es ampliamente conocido por los desarrolladores web entre otras cosas gracias a ser de código abierto y completamente gratuito. 
Contiene una SDK que proporciona un conjunto de APIs de Java que permiten el desarrollo de aplicaciones AJAX escritas en Java. Posteriormente compila el código en JavaScript ya optimizado dando robustez a la aplicación web.

Dentro de las ventajas con las que cuenta esta herramienta reparo en que no hace falta compilar el código JavaScript para poder verlo en el buscador y puedo editar y refrescar la página para ver los cambios. (Faltan cosas)

