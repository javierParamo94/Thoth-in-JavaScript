\capitulo{5}{Aspectos relevantes del desarrollo del proyecto}

Lo más relevante de este proyecto es el transformar la última versión de Thoth a una aplicación web hecha por medio de GWT lo que supone comprender el funcionamiento interno del Thoth original, <<desglosándolo>> para poder adaptarlo a las condiciones de un proyecto hecho con GWT.
Estas condiciones limitan un poco la aplicación y nos obligan a reformar partes que antes eran más sencillas.

Es el caso, por ejemplo del núcleo de la gramática de Thoth que no puede ser llevado a un proyecto en GWT tal cual y necesita ser adaptado para que funcione como en la aplicación original.




\section{Internacionalización}
La aplicación cuenta con la funcionalidad de la internacionalización. Dentro del menú se pueden elegir entre varios idiomas a los que se traducirán los diferentes elementos. Los idiomas en los que está disponible la versión web de Thoth son: Alemán, castellano o español, francés y por supuesto inglés. Consideramos que esta funcionalidad es muy importante para poder llegar a diferentes países en el caso de que fuera necesario.

La internacionalización de la aplicación es un poco diferente a la utilizada en la versión de escritorio de Thoth. En primer lugar es necesario incluir una interfaz con los métodos para la internacionalización y los mensajes <<por defecto>> asociados a cada uno. Cada vez que queramos hacer uso de esos mensajes hay que hacer una llamada al método de la interfaz. Esa interfaz se encuentra en el directorio <<client.gui.utils>> donde se encuentran también los ficheros <<properties>> asociados, donde se encuentran las diferentes traducciones según el mensaje. Estos mensajes son los mismos que los utilizados para la internacionalización de Thoth V2.

Para poder realizar el cambio de idioma es necesario hacer uso de las propiedades las clases <<xml>> y <<html>> de GWT, en concreto <<locale>> que es la que especificará la localidad, que determina el idioma. Por ello cada vez que elegimos un idioma, la aplicación se redirige a una nueva <<URL>> (llevando a cabo una nueva compilación) con el atributo <<locale=>> seguido de las siglas del idioma al que se quiere traducir.


documentar gwt 
 
 
 
 documentar algo del cliente serivido, que me ha costado etc.



\section{Registro e inicio de sesión de usuarios}
Todas las comunicaciones rpc, el porque utilizo modulos de carga, porque dan errores si se encuantran los modulos en las variables a ejecutar el programa.
Sobre la session y la cookies, como utilizo el gwt.xml de app engine y porque da fallo
